\section{Resource Management}

Since we chose Laravel as our MVC framework, before we could define a resource management system in our project, we first needed a database.
Our database solution of choice ended up being Supabase with PostgreSQL. We settled on it because it is very easy to set up, provides a generous
free plan, has an intuitive UI, and can be easily scaled up by upgrading the database plan.\\
With the database set up, we created a schema for our application using draw.io (the ERD diagram can be found in the Appendix) and started writing Laravel 
migrations to implement the database. The central two models of our database are User and Set (stored in \textit{users} and \textit{sets} tables respectively)
, the latter storing all of the most important data about a LEGO set.
The rest of the models within our database center around adding additional typechecking or information to the sets, such as the set's price records.\\
As of the current state of the project, we populate the database from multiple different sources, and using multiple tools, but this could be streamlined 
if this application were to evolve.
\begin{itemize}
    \item \textbf{Seeders} - We defined some basic database seeders for including static data such as the testing admin account (to be removed for production) or set availability types
    \item \textbf{Python Web Scraper} - A simple python console program utilizing \textit{Selenium} to scrape Brickeconomy for set themes and subthemes
    \item \textbf{Playwright Web Scraper} - The main tool for obtaining the current prices of the sets within our database, an implementation of \textit{Playwright} that scrapes eBay for price records
    \item \textbf{Admin Upload Data Page} - Albeit a temporary solution, this is currently the main tool for adding new sets into the database 
\end{itemize}
Adhering to the MVC framework, we hydrate our views with data by the use of Laravel Controllers. Almost every single page has its own controller, so
that the data can be custom formatted and optimized to the needs of that specific page. Our controllers perform different CRUD operations, and some of them can
only be accessed by the admin user, as to comply with the project's requirements of user roles and application functionality. The operations include:
\begin{itemize}
    \item \textbf{Account CRUD} - Before a user is logged in (checked by Laravel Breeze), they are only able to see our landing page with the ability to create an account.
    Most of the account management functionality was already pre-provided for us by Laravel Breeze, which made the development process a lot smoother,
    as we had to either user the pre-existing controllers or recycle their functions. Most note-worthy, the \textit{RegisteredUserController} manages user account creation,
    and the \textit{PasswordController} manages user authentication. The rest of the controllers within the Auth directory are used within the Settings page
    for editing the account details and deleting the account itself.
    \item \textbf{Set Details CRUD} - The admin-only Upload Data page allows the admin to upload CSV files that contain information about the set they want
    to add into the database. We created a small, initial dataset for our application, since we knew adding sets would require significant moderation. The data
    sanitization and creation is handled by the \textit{FileUploadController}.
    \item \textbf{User Favourites CRUD} - In the settings page, a user can select their favourite sets, themes and subthemes from all the available ones
    in the database. The controller responsible for this functionality is \textit{SettingsController}.
    \item \textbf{User Inventory CRUD} - From the settings page, a user is able to add a set to their own set Inventory, where they can see a summary
    of all the sets they own. This is handled by the \textit{InventoryController}.
    \item \textbf{User Dashboard Layout CRUD} - The \textit{DashboardController} is responsible for both providing the data for the dashboard view, as well
    as saving the user custom created dashboard within the Edit Dashboard Layout page.
\end{itemize}