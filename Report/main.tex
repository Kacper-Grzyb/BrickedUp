\documentclass[letterpaper,twocolumn]{article}

\usepackage{listings}
\usepackage{pdflscape}
\usepackage{booktabs}
\usepackage{hyperref}
\usepackage{url}


\newcommand{\myparagraph}[1]{\vspace{0.1cm}\noindent \textbf{\textit{#1.}}}

\title{Bricked Up}
\author{Francesco Schenone \and Sebestyen Deak \and Kacper Grzyb \and Ignad Bozhinov \and Leonardo Gianola}
\date{17-12-2024}

\begin{document}

\maketitle

\section{Introduction}

The LEGO investment market has emerged as a profitable opportunity, with an average return of 11\% \cite{lego_market}. However, the current process for analyzing prices, trends, and data on LEGO sets needs to be more cohesive and efficient.  
To address this, \textit{Bricked Up} centralizes key information, providing interactive graphs, historical price comparisons, and tools to explore investment opportunities efficiently. This report outlines the development of the software, inspired by the Bloomberg Terminal, and its role in simplifying LEGO set investments for users.


\section{Front-End}

Before beginning to develop all of our views, as any respectable software engineer would do, we set out to create a prototype. 
As is industry standard, we used Figma to create our prototype, which included all the views we were planning to make and a general 
outline on how the application should look. Our intention was to use a component library both for the Figma prototype as well as 
during development, but given our unique design language it did not come to be.
Since we were building a platform akin to Bloomberg Terminal (but for lego), we inspired our views by it, by using that color palette
and a similar design language: very information dense, dark-mode only and a terminal-like view. Once the Figma prototype was developed, 
we reviewed it, and started developing the front-end.
\\

As our framework of choice ended being Laravel, we tried to adopt as many Laravel defaults as possible, which was reflected 
in the way we developed our front-end. 
As templating engine, we used blade, a simple yet effective templating engine which allows for mix-and-matching HTML with PHP, 
dividing the application into components for a greater reusability and modularity. Blade is the default templating engine in Laravel, 
which meant we did not had to spend any time setting the project up but could go right to develop the views of our web app. 
We also ended up using alpine.js \cite{alpinejs} in some parts of the application, when handling graphs or to create the scrolling bar with the prices for example, where vanilla Javascript was turning out to be too complicated
or where inline PHP simply was not enough. 

Many components were divided in reusable blade components, such as the navbar, all the elements of the dashboard, as to allow customization of such, 
the authentication components and other shared items. By making use of the components we greatly reduced repeated code, allowing us to adhere as much as possible 
to the DRY \cite{dry} principle. As an upside it also sped up development time considerably, as each team member could work on a component of the website indipendently, 
without needing to wait for pages to be complete yet. 

The HTML we produced was heavily subdivided in with various \texttt{div} elements. That allowed again for a greater modularity and enabled to easily style 
different components in our web app with CSS. Most \texttt{div} element had a class associated with them, which then we would separatily style with a CSS document. 
We did not use a premade CSS stylesheet, but wrote our own, to adhere to our unique design language. All of the CSS was kept in the default \texttt{app.css}. Given the extensive use
of \texttt{div} elements with distinct classes, no conflicts emerged between components. 
Other HTML tags that were used are the \texttt{p} to hold text, \texttt{li} and \texttt{ul} tags to create lists and \texttt{img} tags to hold images correctly. 

The appropriate use of Javascript, modern HTML and CSS together with a powerful templating engine like Blade, allowed for a visually pleasing end product,
with what we believe to be an intuitive UI and UX. 

\myparagraph{Resources} Lectures 1 to 3.

\myparagraph{Length} 2 columns.

\section{Resource Management}

Since we chose Laravel as our MVC framework, before we could define a resource management system in our project, we first needed a database.
Our database solution of choice ended up being Supabase with PostgreSQL. We settled on it because it is very easy to set up, provides a generous
free plan, has an intuitive UI, and can be easily scaled up by upgrading the database plan.\\
With the database set up, we created a schema for our application using draw.io (the ERD diagram can be found in the Appendix) and started writing Laravel 
migrations to implement the database. The central two models of our database are User and Set (stored in \textit{users} and \textit{sets} tables respectively)
, the latter storing all of the most important data about a LEGO set.
The rest of the models within our database center around adding additional typechecking or information to the sets, such as the set's price records.\\
As of the current state of the project, we populate the database from multiple different sources, and using multiple tools, but this could be streamlined 
if this application were to evolve.
\begin{itemize}
    \item \textbf{Seeders} - We defined some basic database seeders for including static data such as the testing admin account (to be removed for production) or set availability types
    \item \textbf{Python Web Scraper} - A simple python console program utilizing \textit{Selenium} to scrape Brickeconomy for set themes and subthemes
    \item \textbf{Playwright Web Scraper} - The main tool for obtaining the current prices of the sets within our database, an implementation of \textit{Playwright} that scrapes eBay for price records
    \item \textbf{Admin Upload Data Page} - Albeit a temporary solution, this is currently the main tool for adding new sets into the database 
\end{itemize}
Adhering to the MVC framework, we hydrate our views with data by the use of Laravel Controllers. Almost every single page has its own controller, so
that the data can be custom formatted and optimized to the needs of that specific page. Our controllers perform different CRUD operations, and some of them can
only be accessed by the admin user, as to comply with the project's requirements of user roles and application functionality. The operations include:
\begin{itemize}
    \item \textbf{Account CRUD} - Before a user is logged in (checked by Laravel Breeze), they are only able to see our landing page with the ability to create an account.
    Most of the account management functionality was already pre-provided for us by Laravel Breeze, which made the development process a lot smoother,
    as we had to either user the pre-existing controllers or recycle their functions. Most note-worthy, the \textit{RegisteredUserController} manages user account creation,
    and the \textit{PasswordController} manages user authentication. The rest of the controllers within the Auth directory are used within the Settings page
    for editing the account details and deleting the account itself.
    \item \textbf{Set Details CRUD} - The admin-only Upload Data page allows the admin to upload CSV files that contain information about the set they want
    to add into the database. We created a small, initial dataset for our application, since we knew adding sets would require significant moderation. The data
    sanitization and creation is handled by the \textit{FileUploadController}.
    \item \textbf{User Favourites CRUD} - In the settings page, a user can select their favourite sets, themes and subthemes from all the available ones
    in the database. The controller responsible for this functionality is \textit{SettingsController}.
    \item \textbf{User Inventory CRUD} - From the settings page, a user is able to add a set to their own set Inventory, where they can see a summary
    of all the sets they own. This is handled by the \textit{InventoryController}.
    \item \textbf{User Dashboard Layout CRUD} - The \textit{DashboardController} is responsible for both providing the data for the dashboard view, as well
    as saving the user custom created dashboard within the Edit Dashboard Layout page.
\end{itemize}

\section{Authentication and Authorization}

This project's third and final task is to incorporate authentication and authorization capabilities.
This section should clearly describe:

\begin{itemize}
    \item Authentication: The different users of the system and how it is implemented
    \item Authorization: Summarize the access of the different users in the system and how it is implemented
    \item Role table: Include a role table associating actions over the system (you can think of them as use cases) and users that can perform these actions.
\end{itemize}

\myparagraph{Resources} Lecture 7.

\myparagraph{Length} 2 columns.

\section{Conclusions}

As said in the Introduction, in our Web Technologies Project we aimed to aid the people who want to get into investing in the world of LEGO. 
Achieving this required us to build and develop a system that can track and display the prices and trends of this market and show it to the user through our intuitive UI.
We worked in PHP with the Laravel Framework, utilizing different libraries and Javascript for ease of use and a better experience. \n
The heart of the application just reads from the database and using the data it shows the bare prices of sets, some information about them and basic graphs. 
This itself wouldn't be special, but the individual extencions take it to a whole another level.


\textbf{Individual Extention Topics:}
\begin{itemize}
    \item Graphs and Alerts: Intuitive, Trading-Like Graphs and the possibility to get notified about price changes.
    \item Two-Factor Authorization: Better security for the User's profile.
    \item Web Scraper: Allowing the application to not only work with a static database, but update it when needed.
    \item Dashboard Customization: Enabling the user to create the Home Page that perfectly fits them.
    \item User Inventory and AJAX Requests: Giving the user the ability to follow their favourite sets, and upgrading the UI to be more responsive
\end{itemize}





\begin{thebibliography}{9}
    
    \bibitem{lego_market}
    Dmitry B. Krylov, "LEGO investing as an alternative asset class: Annualized returns of 11\%," 
    \textit{ScienceDirect}, 2021. Available at: \href{https://www.sciencedirect.com/science/article/abs/pii/S0275531921001604?via%3Dihub}{ScienceDirect Article}.
    \bibitem{alpinejs} \href{https://alpinejs.dev}{Alpine.js}
    \bibitem{dry} \href{https://en.wikipedia.org/wiki/Don%27t_repeat_yourself}{DRY principle}
    
\end{thebibliography}

\end{document}

