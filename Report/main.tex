\documentclass[12pt]{report}
\usepackage{amsmath}
\usepackage{graphicx}
\usepackage{hyperref}
\usepackage[utf8]{inputenc}

\title{Bricked Up}
\author{Francesco Schenone \and Sebestyen Deak \and Kacper Grzyb \and Ignad Bozhinov \and Leonardo Gianola}
\date{17-12-2024}

\begin{document}

\begin{titlepage}
    \centering
    \vspace*{1cm}
    
    \Huge
    \textbf{Syddansk Universitet}
    
    \vspace{2cm}
    
    \Huge
    \textbf{Project Report}
    
    \vfill
    
    \large
    Faculty of Engineering.\\
    BSc in Software Engineering\\
    Course Teacher: Mubashrah Saddiqa\\
    Project period: 2024.09.06 -- 2024.12.17
    
    \vfill
    \begin{flushleft}  
      Group Participants: \\
    \end{flushleft}
      \hfill \small{\textbf{Francesco Schenone} (frsch23@student.sdu.dk)} \\
      \hfill \small{\textbf{Sebestyén Deák} (sedea23@student.sdu.dk)} \\
      \hfill \small{\textbf{Kacper Grzyb} (kagrz23@student.sdu.dk)} \\
      \hfill \small{\textbf{Leonardo Gianola} (legia23@student.sdu.dk)} \\
      \hfill \small{\textbf{Ignat Bozhinov} (igboz23@student.sdu.dk)} \\
  
  \end{titlepage}

% Chapter 1
\chapter{Introduction}
The goal of the introduction is to let the readers (theprofessor and TAs) know the topic of your work andthe main takeaways of it.  The introduction should bebroad enough to understand the document withoutreading it and specific enough to let the reader know:If you are interested in this topic, you should readthis work.In the context of the Web Technologies course, theintroduction should clearly describe:•Motivation:  what problem does this work try tosolve (and why is it important)•Project:   clearly  describes  the  topic  the  groupchose to work with•Contributions:  main takeaways that readers willget from reading this workRemember  to  keep  this  and  all  other  sectionswithin  the  page  and  column  limits.   It  is  your  re-sponsibility to describe first the most important andinteresting aspect of each section.  That way, you canleave  behind  non-interesting  and  repeated  informa-tion more easily.Length.Half a column.

% Chapter 2
\chapter{Front-End}
This project’s first task is to build your application’sfront-end  side.   This  section  should  clearly  describethe  technical  implementation  of  the  work  put  intobuilding the front-end:•Technically describe the use of HTML 5:  whichHTML tags do you use, where, and whyTechnically  describe  the  use  of  CSS:  why  andhow  you  use  CSS  (including  interesting  selec-tors/declarations and how it is incorporated inthe application)•Technically describe the use of JavaScript:  whyand how you use it in your application (includinginteresting behaviors and how they are incorpo-rated into the application)Resources.Lectures 1 to 3.Length.2 columns.

% Chapter 3
\chapter{Resource Management}
The second task of this project is to build a resourcemanagement.  A resource is a model of an object inyour system, and it could be anything:  movies, mu-sic albums, pets, etc.  This section should clearly de-scribe  the  management  of  the  chosen  resource:  theCRUD  (Create-Read-Update-Delete)  operations  as-sociated with that resource:•Technically describe the resource (ie, the model)•Technically describe how the CRUD operationsare implementedResources.Lectures 4 to 6.Length.2 columns.

% Chapter 4
\chapter{Authentication and Authorization}
This project’s third and final task is to incorporateauthentication  and  authorization  capabilities.   Thissection should clearly describe:1
•Authentication:  The  different  users  of  the  sys-tem and how it is implemented•Authorization:  Summarize the access of the dif-ferent users in the system and how it is imple-mented•Role table:  Include a role table associating ac-tions  over  the  system  (you  can  think  of  themas use cases) and users that can perform theseactions.Resources.Lecture 7.Length.2 columns

% Chapter 5
\chapter{Conclusions}
The goal of the conclusion is similar to the introduc-tion: it summarizes the work itself and the takeawaysa reader should take when reading this work.  How-ever, it can use the information presented in the workto be more specific than the introduction.In the context of the Web Technologies course, theconclusion should clearly describe:•Summary:  summary of the work and main take-aways.  Also include a class diagram of the sys-tem (the models).•Future Work:  interesting directions on how thepresented work can evolve in the future (it maybe the starting point to choose individual exten-sion topics)Length.Half a column.

\end{document}
